\chapter{Guida alla lettura}
Questo scritto  è ``a più livelli''. Come noterai sono presenti diverse note a piè di pagina, appendici, ecc\ldots Tutto ciò è un ``di più'', non strettamente necessario a passare l'esame, ma se da subito ti accorgi che l'Informatica ha un suo fascino, potrebbe stimolarti ad ``andare oltre'': qualche piccola curiosità quà e là, qualche approfondimento, qualche chiarimento\ldots

Ovviamente, puoi saltare tutto e andare solo al dunque. 

In appendice, sono presenti una rapida guida per familiarizzare con Linux, una guida per l'installazione di ROOT e qualche tema d'esame risolto.\\


Ci tengo a sottolineare una cosa: queste note non sono appunti del corso o, tanto meno, una ``\emph{dispensa del corso}'' ufficiale. Piuttosto, dopo essermi appassionato all'Informatica e dopo aver approfondito questa materia, ho deciso, anni dopo il corso, di provare a ``buttare giù'' qualche riga. Ti dico tutto ciò per metterti in guardia: ho seguito una mia linea espositiva cercando di coprire tutti gli argomenti del corso, ma lungi da me l'intento di sostituirmi alle lezioni, preziosissime e da frequentare! Confrontati sempre con loro!\\

Infine, così come il corso, queste note sono pensate per essere ``introduttive'' al \verb|C++| e alla programmazione; per cui, a volte, l'esposizione è semplicistica per favorire la comprensione. Sappi, però, che il \verb|C++| è estremamente più vasto, complesso e completo di quanto da queste note possa apparire, citando l'autore del libro che più mi ha istruito\footnote{Thinking in C++, Bruce Eckel}: ``\emph{C} -e, per eredità, il \verb|C++|- \emph{ has never been a language to bar entry where others fear to tread}''. \\

Detto ciò, buona lettura\ldots Ah, se trovi imprecisioni, errori o chissà quale altra castroneria, segnalamela (\href{mailto:lorenzo.uboldi@studenti.unimi.it}{\ttfamily lorenzo.uboldi@studenti.unimi.it}), te ne sarò grato!